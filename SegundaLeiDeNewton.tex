\documentclass[article,12pt,oneside,a4paper,brazil]{abntex2}

\usepackage[alf]{abntex2cite}
\usepackage{graphicx} % Required for inserting images
\usepackage[utf8]{inputenc}
\usepackage[brazil]{babel}
\usepackage[outline]{contour} % glow around text
\usepackage{lipsum}
\usepackage{amsmath} % For align environment
\usepackage{cancel}
\usepackage{blindtext}
\usepackage{tikz, tkz-base, tkz-fct}
\usepackage{pgfplots}
\usepackage{indentfirst}
\usepackage{multirow}
\usepackage{physics}
\usepackage[T1]{fontenc}    % para suporte a caracteres acentuados

\usetikzlibrary{angles,quotes} % for pic
\contourlength{1.2pt}

% Estou colocando 2 de espaço por cada pergunta

% Personalizando as coisas do latex
\usepackage[left=2.5cm,top=2.5cm,right=2.5cm,bottom=2.5cm]{geometry}


\begin{document}
	\begin{flushleft}
		\textbf{RELATÓRIO 06 - A SEGUNDA LEI DE NEWTON}
		
		\textbf{CURSO:} Física - Bacharelado
		
		\textbf{TURMA:} BF2
		
		\textbf{INTEGRANTES:}	
		
		\begin{adjustwidth}{1cm}{0cm}
			\begin{itemize}
				\item Ítalo M. M. F. Leite ;
				\item Chiang Kai Shek dos Santos Macedo.
			\end{itemize}
		\end{adjustwidth}
	\end{flushleft}

	\section{HABILIDADES E COMPETÊNCIAS}
	
	Este experimento tem como objetivo investigar, por meio da coleta de dados, as relações de proporcionalidade entre as grandezas físicas descritas pela Segunda Lei de Newton. Além disso, uma habilidade crucial que será aprimorada é a competência em medir e analisar dados experimentais.
	
	Ao final do experimento, espera-se que os participantes desenvolvam uma maior 	precisão na medição. Isso inclui o uso adequado de equipamentos, como dinamômetros para medir a força aplicada e sensores para medir a aceleração, com o intuito de obter dados precisos e confiáveis. Também se espera que melhorem no registro e na análise de dados, realizando a coleta de informações de forma sistemática e registrando-as corretamente. A análise dos dados envolverá o cálculo da aceleração e a comparação com a força aplicada para verificar a validade da relação $F=m \cdot a$. Além disso, os participantes deverão avaliar se os resultados experimentais estão de acordo com as previsões teóricas e, se não estiverem, identificar possíveis fontes de erro e compreender como esses erros podem ter influenciado os dados obtidos. Os participantes também têm que desenvolver as seguintes comptências
	
	\begin{adjustwidth}{1cm}{0cm}
		\begin{itemize}
			\item Medir e determinar o peso de diversas massas;
			\item Medir a aceleração do móvel;
			\item Relacionar as grandezes força e aceleração;
			\item Definir a segunda lei de Newton.
		\end{itemize}
	\end{adjustwidth}
	
	\section{FUNDAMENTOS TEÓRICOS}
	
	Os princípios fundamentais da dinâmica foram desenvolvidos por Galileu Galilei e Isaac Newton. No entanto, já na Grécia Antiga, filósofos como Aristóteles haviam feito considerações sobre o movimento. Aristóteles, por exemplo, acreditava que um corpo precisava de uma força contínua para manter-se em movimento, uma ideia que parecia confirmar a nossa experiência cotidiana, na qual um objeto em movimento sobre o solo tende a parar se a força de empurrar cessar. De acordo com Nussenzveig (2013, p. 91):
	
	\begin{adjustwidth}{4cm}{0cm}
		[...] Segundo Aristóteles, para um corpo se manter em movimento é necessário
		que exista uma força aplicada ao objeto. Isto parece concordar com nossa experiência	imediata de que um objeto deslizando sobre o solo, por exemplo, tende a parar se pararmos de empurrá-lo.
	\end{adjustwidth}

	Adicionalmente, Aristóteles afirmava que um objeto mais pesado cairia mais rapidamente do que um objeto mais leve. Essa visão foi amplamente aceita até o final do século XVII, apesar de haver algumas objeções. Galileu Galilei, por outro lado, desafiou essa concepção, influenciado pelas ideias de Nicolau Copérnico sobre o movimento planetário. Galileu enfrentou grandes dificuldades para superar as ideias aristotélicas, especialmente em relação ao movimento de queda livre. Ele desenvolveu a teoria da inércia e concluiu que, em um sistema ideal sem resistência do ar, um corpo em queda livre se moveria com uma velocidade constante, contradizendo a visão aristotélica.
	
	A ideia de Galileu era difícil de ser testada com a tecnologia da época. Apenas séculos depois, com o avanço tecnológico, foi possível criar Tubos de Vácuo em torno de 1905, permitindo a comprovação de suas teorias.
	
	A noção de sistema de referência é crucial na mecânica clássica. Em particular, um
	sistema de referência que respeita todas as leis do movimento é conhecido como sistema de referência inercial.
	
	Os princípios da dinâmica foram formalizados por Isaac Newton em 1687, em sua obra
	monumental Philosophiæ Naturalis Principia Mathematica. Newton formulou as três leis do movimento, conhecidas como Leis de Newton. Embora Newton não tenha realizado os experimentos diretamente, suas leis foram deduzidas com base em trabalhos prévios de cientistas como Galileu Galilei, Johannes Kepler e Robert Hooke e etc.
	
	A primeira lei da dinâmica, também conhecida como Lei da Inércia, afirma, de acordo
	com Nussenzveig (2013, p. 93): “Todo corpo persiste em seu estado de repouso ou de
	movimento retilíneo uniforme, a menos que seja compelido a modificar esse estado pela ação de forças impressas sobre ele.”

	A segunda lei de Isaac Newton é o princípio fundamental da dinâmica; ela permite
	solucionar a igualdade de que a força resultante sobre um corpo é igual a razão da variação do momento linear sobre sua variação do tempo. A formulação geral da segunda lei, como expressa a equação seguinte:
	
	\begin{equation*}
		F = \frac{d\vec{p}}{dt}
	\end{equation*}

	Onde $F$ é a força resultante do corpo, $dp$ é a variação do momento linear e $dt$ é a variação do tempo. O momento linear é o produto do peso do objeto com sua velocidade, ou seja, a fórmula pode ser descrita da seguinte forma:

	\begin{equation*}
		F = \frac{d(m\cdot\vec{v})}{dt}
	\end{equation*}
	
	Como a massa não varia durante o tempo, podemos tira-la da derivada e derivar a velocidade isoladamente.
	
	\begin{equation*}
		F = m \cdot \frac{d\vec{v}}{dt}
	\end{equation*}
	
	Sabemos que a derivada da velocidade da a aceleração, pois a derivada é a taxa de variação instantânea de um móvel qualquer. Onde a aceleração do objeto é constante.
	
	\begin{equation*}
		F = m \cdot a
	\end{equation*}
	
	\section{MATERIAIS E MÉTODOS}

	\subsection{Materiais Utilizados:}
	\begin{adjustwidth}{1cm}{0cm}
		\begin{itemize}
			\item 01 - Trilho de ar, que gera um fluxo de ar;
			\item 01 - Conjunto de hastes paralelas;
			\item 01 - Um móvel que desliza sobre o trilho (carrinho);
			\item 01 - Cerca ativadora com 10 intervalos de 18 milimetros;
			\item 01 - Gancho curto;
			\item 01 - Sensor fotoelétrico e cabo Mini-Din;
			\item 08 - Massas acoplável de 10,0 + 0,1 g;
			\item 01 - Uma corda pequena e fina;
			\item 01 - Dinamômetro com intervalos de 2N cada troca de cor;	
			\item 01 - Computador com Software Libre OfficeCalc.
	\end{itemize}
	\end{adjustwidth}

	\subsection{Preparação e Execução do Experimento:}
	
	Precedendo a execução do experimento, alinhamos o carrinho junto ao sensor
	fotoelétrico. Ativamos o sensor fotoelétrico, configurando-o na função F3, que registra os tempos dos dez intervalos de 0,018 metros correspondentes à passagem da cerca ativadora. Inicialmente, colocamos os 8 pesos sobre o carrinho e deixamos o gancho sem peso.
	
	Para realizar o experimento, ligaremos o gerador de fluxo e seguraremos o carrinho com a mão, soltando-o assim que e o sensor fotoelétrico estiverem em operação. Repetiremos o experimento 8 vezes, adicionando uma massa acoplável ao gancho curto a cada repetição, aumentando assim o peso. Após cada adição de massa, utilizaremos o dinamômetro para medir o peso total do gancho com as massas acopladas.
		
	\subsection{Coleta de Dados}
	
	Registramos os tempos entre os intervalos coletados pelo sensor fotoelétrico da cerca ativadora e o peso coletado no dinamômetro no gancho curto. O \textit{Software LibreOffice} foi utilizado para fazer a tabela de comparação entre os intervalos de tempo em relação a força. Com os dados coletados do tempo em relação a força, utilizamos as deduções das equações feitas no fundamentos teóricos para efetuar as contas e descobrir a aceleração do carrinho em cada uma das condições.
	
	\section{RESULTADO E ANÁLISE DE DADOS}
	A seguir apresentamos a Tabela 1 com dados de posição x em função do tempo t em
	segundos obtidos durante a realização do experimento da segunda lei de Newton com o trilho de ar, dinamômetro e os pessos acoplados. Verificando a validade das equações de $F=m \cdot a$ da segunda lei de Newton.
	

	\begin{table}[h!]
		\centering
		\begin{adjustwidth}{0cm}{0cm}
			\begin{flushleft}
				\textbf{Tabela 1 -} Intervalos de tempo em relação à força.
			\end{flushleft}
		\end{adjustwidth}
	
		\label{tab:intervalos-forca}
		\begin{tabular}{|c|c|c|c|c|c|c|c|c|}
			\hline
			& \textbf{F1 (N)} & \textbf{F2 (N)} & \textbf{F3 (N)} & \textbf{F4 (N)} & \textbf{F5 (N)} & \textbf{F6 (N)} & \textbf{F7 (N)} & \textbf{F8 (N)} \\ \hline
			\textbf{F aplicada} & 0,18 & 0,28 & 0,38 & 0,48 & 0,56 & 0,72 & 0,76 & 0,86 \\ \hline
			\textbf{x (m)} & \textbf{t (s)} & \textbf{t (s)} & \textbf{t (s)} & \textbf{t (s)} & \textbf{t (s)} & \textbf{t (s)} & \textbf{t (s)} & \textbf{t (s)} \\ \hline
			% Linha 1
			0 & 0 & 0 & 0 & 0 & 0 & 0 & 0 & 0 \\ \hline
			% Linha 2
			0,018 & 0,25615 & 0,195 & 0,1687 & 0,14345 & 0,14055 & 0,1342 & 0,12110 & 0,10645 \\ \hline
			% Linha 3
			0,036 & 0,3797 & 0,29435 & 0,254 & 0,21845 & 20955 & 0,1983 & 0,1809 & 0,16175 \\ \hline
			% Linha 4
			0,054 & 0,47475 & 0,37055 & 0,31935 & 0,27615 & 0,26235 & 0,2473 & 0,22665 & 0,20445 \\ \hline
			% Linha 5
			0,072 & 0,55485 & 0,4346 & 0,3745 & 0,32505 & 0,30705 & 0,28875 & 0,2653 & 0,24065 \\ \hline
			% Linha 6
			0,09 & 0,6256 & 0,4912 & 0,42305 & 0,36825 & 0,34645 & 0,3253 & 0,29925 & 0,27235 \\ \hline
			% Linha 7
			0,108 & 0,68885 & 0,542 & 0,46685 & 0,407 & 0,3817 & 0,35795 & 0,3298 & 0,30095 \\ \hline
			% Linha 8
			0,126 & 0,7478 & 0,58915 & 0,5074 & 0,44295 & 0,4145 & 0,3883 & 0,35835 & 0,3277 \\ \hline
			% Linha 9
			0,144 & 0,8025 & 0,63295 & 0,54505 & 0,4765 & 0,445 & 0,4166 & 0,38465 & 0,35235 \\ \hline
			% Linha 10
			0,162 & 0,85355 & 0,6739 & 0,58025 & 0,5077 & 0,47335 & 0,4429 & 0,40910 & 0,3753 \\ \hline
			% Linha 11
			0,80 & 0,9022 & 0,7129 & 0,61375 & 0,5375 & 0,5005 & 0,468 & 0,43270 & 0,3974 \\ \hline
		\end{tabular}
		
		\begin{adjustwidth}{0cm}{0cm}
			\begin{flushleft}
				\textbf{Fonte: }Autoria Própria
			\end{flushleft}
		\end{adjustwidth}
	\end{table}

	O Gráfico 1 foi feito com os dados da Tabela 1. O gráfico do espaço em relação ao tempo.
	
	\begin{center}
		\begin{figure}[h!]
			\centering
			\begin{flushleft}
				\textbf{Gráfico 1 -} Espaço em espaço em relação ao tempo.
			\end{flushleft}
			\begin{tikzpicture}
				\begin{axis}[
					width=\textwidth, % Largura do gráfico
					height=0.6\textwidth, % Altura do gráfico
					xlabel={$t$ (s)},
					ylabel={$x(t)$ (m)},
					legend pos=north west,
					grid=both,
					title={Gráfico da Tabela 1},
					legend style={font=\small}
					]
					
					% Dados para F1
					\addplot[blue, mark=square*] coordinates {(0.25615, 0.018) (0.3797, 0.036) (0.47475, 0.054) (0.55485, 0.072) (0.6256, 0.09) (0.68885, 0.108) (0.74778, 0.126) (0.8025, 0.144) (0.85355, 0.162) (0.9022, 0.18)};
					\addlegendentry{$f_1$}
					
					% Dados para F2
					\addplot[red, mark=*] coordinates {(0.195, 0.018) (0.29435, 0.036) (0.37055, 0.054) (0.4346, 0.072) (0.4912, 0.09) (0.542, 0.108) (0.58915, 0.126) (0.63295, 0.144) (0.6739, 0.162) (0.7129, 0.18)};
					\addlegendentry{$f_2$}
					
					% Dados para F3
					\addplot[green, mark=triangle*] coordinates {(0.1687, 0.018) (0.254, 0.036) (0.31935, 0.054) (0.3745, 0.072) (0.42305, 0.09) (0.46885, 0.108) (0.51175, 0.126) (0.54505, 0.144) (0.58025, 0.162) (0.61375, 0.18)};
					\addlegendentry{$f_3$}
					
					% Dados para F4
					\addplot[orange, mark=diamond*] coordinates {(0.14345, 0.018) (0.21845, 0.036) (0.27615, 0.054) (0.32505, 0.072) (0.36825, 0.09) (0.407, 0.108) (0.44245, 0.126) (0.4765, 0.144) (0.5077, 0.162) (0.5375, 0.18)};
					\addlegendentry{$f_4$}
					
					% Dados para F5
					\addplot[purple, mark=star] coordinates {(0.14055, 0.018) (0.20935, 0.036) (0.26235, 0.054) (0.3065, 0.072) (0.34615, 0.09) (0.3817, 0.108) (0.41455, 0.126) (0.445, 0.144) (0.47335, 0.162) (0.5005, 0.18)};
					\addlegendentry{$f_5$}
					
					% Dados para F6
					\addplot[cyan, mark=o] coordinates {(0.1342, 0.018) (0.1983, 0.036) (0.2473, 0.054) (0.28875, 0.072) (0.32545, 0.09) (0.35795, 0.108) (0.3883, 0.126) (0.4166, 0.144) (0.4429, 0.162) (0.468, 0.18)};
					\addlegendentry{$f_6$}
					
					% Dados para F7
					\addplot[magenta, mark=+] coordinates {(0.1211, 0.018) (0.1809, 0.036) (0.22665, 0.054) (0.26635, 0.072) (0.29925, 0.09) (0.3298, 0.108) (0.3583, 0.126) (0.38465, 0.144) (0.40915, 0.162) (0.4327, 0.18)};
					\addlegendentry{$f_7$}
					
					% Dados para F8
					\addplot[brown, mark=x] coordinates {(0.10645, 0.018) (0.16175, 0.036) (0.20445, 0.054) (0.24065, 0.072) (0.27245, 0.09) (0.30095, 0.108) (0.32755, 0.126) (0.35235, 0.144) (0.3753, 0.162) (0.3974, 0.18)};
					\addlegendentry{$f_8$}
				\end{axis}
			\end{tikzpicture}
			\label{fig:grafo-tabela1}
		\end{figure}
	\end{center}

	\begin{flushleft}
		\textbf{Questionários: }
		
		\textbf{4.9 -} Utilizando as equações das curvas da tendência dos gráficos x versus t, podemos derivar a função para achar a função da velocidade. Segundo os fundamentos teóricos, a derivada da velocidade é a aceleração. Ou seja, a fórmula da derivada é da seguinte forma:	
		
		\begin{equation*}
			a = \dfrac{d^2x}{dt^2}
		\end{equation*}
	
		Derivando todas as funções encontradas na linha tendência irá resultar exatamente na
		aceleração. A Tabela 2 mostra os resultados obtidos:
		
		\begin{center}
			\begin{table}[h!]
				\centering
				\begin{adjustwidth}{0cm}{0cm}
					\begin{flushleft}
						\textbf{Tabela 2}
					\end{flushleft}
				\end{adjustwidth}
				
				\label{tab:intervalos-forca}
				\begin{tabular}{|c|c|}
					\hline
					\textbf{F (N)} & a($m/s^2$)\\ \hline
					% Linha 1
					0,18 & 0,40 \\ \hline
					% Linha 2
					0,028 & 0,61 \\ \hline
					% Linha 3
					0,036 & 0,84 \\ \hline
					% Linha 4
					0,054 & 1,08 \\ \hline
					% Linha 5
					0,072 & 1,30 \\ \hline
					% Linha 6
					0,09 & 1,52 \\ \hline
					% Linha 7
					0,108 & 1,74 \\ \hline
					% Linha 8
					0,126 & 1,96 \\ \hline
				\end{tabular}
				
				\begin{adjustwidth}{0cm}{0cm}
					\begin{flushleft}
						\textbf{Fonte: }Autoria Própria
					\end{flushleft}
				\end{adjustwidth}
			\end{table}
		\end{center}
	
		\textbf{4.10 -} Gráfico 2 é o gráfico da função $F$ versus $a$. O significado físico do coeficiente angular obtido no Gráfico F versus a é a massa da objeto. Pois quanto maior for a aceleração, maior será a força.
		
		\begin{center}
			\begin{figure}[h!]
				\centering
				\begin{flushleft}
					\textbf{Gráfico 2 -} Força em função da aceleração.
				\end{flushleft}
				\begin{tikzpicture}
					\begin{axis}[
						width=\textwidth, % Largura do gráfico
						height=0.6\textwidth, % Altura do gráfico
						xlabel={$t$ (s)},
						ylabel={$x(t)$ (m)},
						legend pos=north west,
						grid=both,
						title={Gráfico 2 da Tabela 2},
						legend style={font=\small}
						]
						
						% Dados para F1
						\addplot[blue, mark=square*] coordinates {(0.018, 0.40) (0.028, 0.61) (0.036, 0.84) (0.054, 1.08) (0.072, 1.30) (0.09, 1.52) (0.108, 1.74) (0.126, 1.96)};
						\addlegendentry{$f_1$}
					\end{axis}
				\end{tikzpicture}
				\label{fig:grafo-tabela2}
			\end{figure}
		\end{center}
	
		\textbf{4.11 -} A velocidade do carrinho aumentou, pois havia mais peso para puxar o carrinho.
		
		\textbf{4.12 -} A afirmação está correta.
	\end{flushleft}

	\section{CONCLUSÃO}
	
	Após a realização dos experimento e a coleta de dados, foi possível verificar a realçao entre a força aplicada e a aceleração do carrinho, conforme previsto pela Segunda Leite de Newton $(F = m \cdot a)$. Os dados coletados através do sensor fotoelétrico permitiram calcular a aceleraçaão do carrinho em cada uma das condições testadas, enquanto o dinamômetro forneceu medições da força aplicada para cada adição de massa ao gancho curto. 
	
	Os resultados obtidos foram registrados e organizados em uma tabela utilizando o 
	\textit{software LibreOffice Calc}, permitindo uma visualização clara das relações entre as grandezas medidas. Com base nos tempos registrados entre os intervalos da cerca ativadora e as forças correspondentes, foi possível calcular a aceleração em cada caso.
	
	Ao analisar os dados, foi observado que, conforme esperado, a aceleração do carrinho
	aumentou proporcionalmente ao aumento da força aplicada, confirmando a segunda lei de Newton. A linearidade dos dados, representada nas tabelas, mostrou uma correlação direta entre a força aplicada e a aceleração, com uma inclinação correspondente à massa do sistema.
	
	No entanto, algumas discrepâncias entre os valores experimentais e os valores teóricos foram observadas. Essas discrepâncias podem ser atribuídas a diversos fatores, como erros de medição, atrito residual no trilho de ar (apesar de ser considerado desprezível), ou pequenas variações na força aplicada devido a inconsistências na soltura manual do carrinho. Esses erros são normais em experimentos físicos e foram considerados na análise dos resultados.
	
	A análise final indicou que os resultados experimentais estavam, em sua maioria, em
	consonância com as previsões teóricas, reforçando a validade da Segunda Lei de Newton. A precisão dos dados foi adequada, e as variações observadas, embora presentes, não comprometeram a conclusão de que a força aplicada é diretamente proporcional à aceleração produzida.
	
	Por fim, os participantes foram capazes de identificar e compreender as possíveis fontes de erro e como elas poderiam ter influenciado os resultados. Essa reflexão é fundamental para o desenvolvimento de habilidades experimentais avançadas, permitindo uma melhor interpretação dos dados em futuros experimentos e um entendimento mais profundo dos conceitos físicos envolvidos.
	
	\section{REFERÊNCIAS BIBLIOGRÁFICAS}
	
	YOUNG, Hugh D.; FREEMAN, Roger A. \textbf{Física 1: Mecânica}. 12. ed. São Paulo: Editora Pearson, 2014.
	
	NUSSENZVEIG, Moysés. \textbf{Curso de Física Básica}. Vol 1. Rio de Janeiro: Editora Blucher, 2013.
\end{document}
